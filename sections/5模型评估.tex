\section{模型评价}
    \subsection{模型优点}
        \begin{enumerate}
            \item 模型一建立了可视化流程图,清晰易懂
            \item 模型二的评价综合性强,为问题五的解决奠定了坚实的基础
            \item 我们建立的最终的调控温度模型所使用的变量少,且
                  调控温度的效果十分明显。
            \item 温度可以进行双向调节,上升下降均可。而且可以对指定
                  区间的温度进行细致调节。
        \end{enumerate}
    \subsection{模型缺点}
        \begin{enumerate}
            \item 在逐步回归和确定变量的过程中,可能会忽略掉一些重要的
                  变量,略微降低了准确度。
            \item 最终调控温度的模型需要确定若干个调节变量,
                  带有一定的主观性,所以可能不是最佳方案
        \end{enumerate}
    \subsection{灵敏度分析}
    \subsection{模型推广}
        \begin{enumerate}
            \item 对于已经建立的管道温度模型,我们可以推广到一般情况的
                  温度调控上,例如锅炉温度调节,冷却水温调节等。也可以
                  判断一般情况的温度影响因素,找到影响温度比 重最大的因
                  子。
        \end{enumerate}