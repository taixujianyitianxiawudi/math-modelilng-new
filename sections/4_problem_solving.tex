\section{绘制普通三线表格}
表格应具有三线表格式,因此常用 booktabs宏包,其标准格式如\cref{tab:001}~所示。
\begin{table}[!htbp]
    \caption{标准三线表格}\label{tab:001} \centering
    \begin{tabular}{ccccc}
        \toprule[1.5pt]
        $D$(in) & $P_u$(lbs) & $u_u$(in) & $\beta$ & $G_f$(psi.in)\\
        \midrule[1pt]
        5 & 269.8 & 0.000674 & 1.79 & 0.04089\\
        10 & 421.0 & 0.001035 & 3.59 & 0.04089\\
        20 & 640.2 & 0.001565 & 7.18 & 0.04089\\
        \bottomrule[1.5pt]
    \end{tabular}
\end{table}

其绘制表格的代码及其说明如下。
\begin{tcode}
    \begin{table}[!htbp]
        \caption[标签名]{中文标题}
        \begin{tabular}{cc...c}
            \toprule[1.5pt]
            表头第1个格   & 表头第2个格   & ... & 表头第n个格  \\
            \midrule[1pt]
            表中数据(1,1) & 表中数据(1,2) & ... & 表中数据(1,n)\\
            表中数据(2,1) & 表中数据(2,2) & ... & 表中数据(2,n)\\
            ...................................................\\
            表中数据(m,1) & 表中数据(m,2) & ... & 表中数据(m,n)\\
            \bottomrule[1.5pt]
        \end{tabular}
    \end{table}
\end{tcode}

\bigskip

table环境是一个将表格嵌入文本的浮动环境。tabular环境的必选参数由每列对应一个格式字符所组成:c表示居中,l表示左对齐,r表示右对齐,其总个数应与表的列数相同。此外,\verb|@{文本}|可以出现在任意两个上述的列格式之间,其中的文本将被插入每一行的同一位置。表格的各行以\verb|\\|分隔,同一行的各列则以\&分隔。 \verb|\toprule| 、\verb|\midrule| 和 \verb|\bottomrule| 三个命令是由booktabs宏包提供的,其中 \verb|\toprule| 和 \verb|\bottomrule| 分别用来绘制表格的第一条(表格最顶部)和第三条(表格最底部)水平线, \verb|\midrule| 用来绘制第二条(表头之下)水平线,且第一条和第三条水平线的线宽为 1.5pt ,第二条水平线的线宽为 1pt 。引用方法与图片的相同。
