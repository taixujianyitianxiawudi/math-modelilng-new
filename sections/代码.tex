\begin{appendices}
    \section{代码}
    \subsection{附件一异常数据处理}
        \begin{lstlisting}[language=matlab]
            S = xlsread('C:\Users\mrgus\Desktop\附件2', 'sheet1', 'B3:EX5002');
            %变量1的异常值处理
            bl1 = S(1:237, 1);
            [m, n] = size(bl1);
            ave = mean(bl1); %均值
            sigma = sqrt((bl1' - ave) * (bl1 - ave) / m); %标准差
            fangcha = sigma^2; %方差
            jicha = max(bl1) - min(bl1); %极差

            sx = ave + 3 * sigma; %用于填充的数据与均值和标准差相关
            xx = ave - 3 * sigma;
            ycz = [];
            zcz = [];
            s = 1;
            s1 = 1;

            for i = 1:m %将得到的数据写入新的行列中

                if bl1(i, 1) < xx || bl1(i, 1) > sx
                    ycz(s, 1) = bl1(i, 1);
                    ycz(s, 2) = i;
                    s = s + 1;
                end

                if bl1(i, 1) < sx && bl1(i, 1) > xx
                    zcz(s1, 1) = bl1(i, 1);
                    zcz(s1, 2) = i;
                    s1 = s1 + 1;
                end

            end

            xlswrite('fujian2 gai.xls', S, 'sheet1', 'a1')

        \end{lstlisting}

    \subsection{问题一温度特征描述代码}
        \begin{lstlisting}[language=matlab]
            clc; clear; close all;

            A = xlsread("附件1.xlsx");

            for i = 1:10 %画图
                x = A(:, 1);
                y = A(:, i + 1);
                figure(i);
                plot(x, y);
                xlabel(['样本序号',num2str(i)]);
                ylabel(['管道', num2str(i), '温度(℃)']);
            end

            mean = mean(A(:, 2:11), 1)  %均值
            var = var(A(:, 2:11), 1)    %方差
            std = sqrt(var)             %标准差
            max = max(A(:, 2:11))       %温度最高值
            min = min(A(:, 2:11))       %温度最低值
        \end{lstlisting}

    \subsection{问题二 数据评价}
        \begin{lstlisting}[language=matlab]
            %读取已经处理完毕的数据
            fujian1 = xlsread('C:\Users\mrgus\Desktop\附件1', 'sheet1', 'B2:K5001');
            fangcha = var(fujian1(:, 1:10), 1);
            zuidazhi = max(fujian1(:, 1:10));

            for i = 1:10

                if zuidazhi(i) > 445;
                    zuidazhi(i) = 1000;
                end

            end

            x = [fangcha' zuidazhi'];

            A = max(x) - x;

            [n, m] = size (A);
            B = A ./ repmat (sum(A .* A).^0.5, n, 1);
            disp ('Normalization matrix B = ')
            disp (B);
            D_P = sum ([(B - repmat (max(B), n, 1)).^2], 2).^0.5;
            D_N = sum ([(B - repmat (min(B), n, 1)).^2], 2).^0.5;
            w = D_N ./ (D_P + D_N);
            disp (' 最综得分: ') %显示最后各管道得分
            stand_w = w / sum(w); %标准化数据
            [sorted_w, index] = sort (stand_w, 'descend') %从高到低将数据进行排列
        \end{lstlisting}

    \subsection{问题三 逐步回归算法}
        \begin{lstlisting}[language=matlab]
            clc; clear; close all;
            Y = xlsread('附件1.xlsx', 'sheet1', 'B2:K5001');
            X = xlsread('附件2改.xls');

            %%管道1
            y = Y(:, 1); %第i列
            [beta, se, pval, in1, stats] = stepwisefit(X, y)
            %做逐步回归分析,其中se是标准误差、pval是p值、in为{0,1|0为删去,1为保留}

            XX1 = []; %留下数组

            for j = 1:size(in1, 2) %in1的列数

                if in1(j) == 1
                    XX1 = [XX1, X(:, j)];
                end

            end

            XX1_ = [ones(size(y)), XX1];
            [b1, bint1, r1, rint1, stats1] = regress(y, XX1_)

            %%管道2
            y = Y(:, 2);
            [beta, se, pval, in2, stats] = stepwisefit(X, y)

            XX2 = [];

            for j = 1:size(in2, 2)

                if in2(j) == 1
                    XX2 = [XX2, X(:, j)];
                end

            end

            XX2_ = [ones(size(y)), XX2];
            [b2, bint2, r2, rint2, stats2] = regress(y, XX2_)

            %%管道3
            y = Y(:, 3);
            [beta, se, pval, in3, stats] = stepwisefit(X, y)

            XX3 = [];

            for j = 1:size(in3, 2)

                if in3(j) == 1
                    XX3 = [XX3, X(:, j)];
                end

            end

            XX3_ = [ones(size(y)), XX3];
            [b3, bint3, r3, rint3, stats3] = regress(y, XX3_)

            %%管道4
            y = Y(:, 4);
            [beta, se, pval, in4, stats] = stepwisefit(X, y)

            XX4 = [];

            for j = 1:size(in4, 2)

                if in4(j) == 1
                    XX4 = [XX4, X(:, j)];
                end

            end

            XX4_ = [ones(size(y)), XX4];
            [b4, bint4, r4, rint4, stats4] = regress(y, XX4_)

            %%管道5
            y = Y(:, 5);
            [beta, se, pval, in5, stats] = stepwisefit(X, y)

            XX5 = [];

            for j = 1:size(in5, 2)

                if in5(j) == 1
                    XX5 = [XX5, X(:, j)];
                end

            end

            XX5_ = [ones(size(y)), XX5];
            [b5, bint5, r5, rint5, stats5] = regress(y, XX5_)

            %%管道6
            y = Y(:, 6);
            [beta, se, pval, in6, stats] = stepwisefit(X, y)

            XX6 = [];

            for j = 1:size(in6, 2)

                if in6(j) == 1
                    XX6 = [XX6, X(:, j)];
                end

            end

            XX6_ = [ones(size(y)), XX6];
            [b6, bint6, r6, rint6, stats6] = regress(y, XX6_)

            %%管道7
            y = Y(:, 7);
            [beta, se, pval, in7, stats] = stepwisefit(X, y)

            XX7 = [];

            for j = 1:size(in7, 2)

                if in7(j) == 1
                    XX7 = [XX7, X(:, j)];
                end

            end

            XX7_ = [ones(size(y)), XX7];
            [b7, bint7, r7, rint7, stats7] = regress(y, XX7_)

            %%管道8
            y = Y(:, 8);
            [beta, se, pval, in8, stats] = stepwisefit(X, y)

            XX8 = [];

            for j = 1:size(in8, 2)

                if in8(j) == 1
                    XX8 = [XX8, X(:, j)];
                end

            end

            XX8_ = [ones(size(y)), XX8];
            [b8, bint8, r8, rint8, stats8] = regress(y, XX8_)

            %%管道9
            y = Y(:, 9);
            [beta, se, pval, in9, stats] = stepwisefit(X, y)

            XX9 = [];

            for j = 1:size(in9, 2)

                if in9(j) == 1
                    XX9 = [XX9, X(:, j)];
                end

            end

            XX9_ = [ones(size(y)), XX9];
            [b9, bint9, r9, rint9, stats9] = regress(y, XX9_)

            %%管道10
            y = Y(:, 9);
            [beta, se, pval, in10, stats] = stepwisefit(X, y)

            XX10 = [];

            for j = 1:size(in10, 2)

                if in10(j) == 1
                    XX10 = [XX10, X(:, j)];
                end

            end

            XX10_ = [ones(size(y)), XX10];
            [b10, bint10, r10, rint10, stats10] = regress(y, XX10_)

            %%构造B
            B = zeros(10, 153)

            n = 2;

            for j = 1:153 %b1排序

                if in1(j) == 1
                    B(1, j) = b1(n);
                    n = n + 1;
                else
                    B(1, j) == 0;
                end

            end

            n = 2;

            for j = 1:153 %b2排序

                if in2(j) == 1
                    B(2, j) = b2(n);
                    n = n + 1;
                else
                    B(2, j) == 0;
                end

            end

            n = 2;

            for j = 1:153 %b3排序

                if in3(j) == 1
                    B(3, j) = b3(n);
                    n = n + 1;
                else
                    B(3, j) == 0;
                end

            end

            n = 2;

            for j = 1:153 %b4排序

                if in4(j) == 1
                    B(4, j) = b4(n);
                    n = n + 1;
                else
                    B(4, j) == 0;
                end

            end

            n = 2;

            for j = 1:153 %b5排序

                if in5(j) == 1
                    B(5, j) = b5(n);
                    n = n + 1;
                else
                    B(5, j) == 0;
                end

            end

            n = 2;

            for j = 1:153 %b6排序

                if in6(j) == 1
                    B(6, j) = b6(n);
                    n = n + 1;
                else
                    B(6, j) == 0;
                end

            end

            n = 2;

            for j = 1:153 %b7排序

                if in7(j) == 1
                    B(7, j) = b7(n);
                    n = n + 1;
                else
                    B(7, j) == 0;
                end

            end

            n = 2;

            for j = 1:153 %b8排序

                if in8(j) == 1
                    B(8, j) = b8(n);
                    n = n + 1;
                else
                    B(8, j) == 0;
                end

            end

            n = 2;

            for j = 1:153 %b9排序

                if in9(j) == 1
                    B(9, j) = b9(n);
                    n = n + 1;
                else
                    B(9, j) == 0;
                end

            end

            n = 2;

            for j = 1:153 %b10排序

                if in10(j) == 1
                    B(10, j) = b10(n);
                    n = n + 1;
                else
                    B(10, j) == 0;
                end

            end

            %%画图
            for i = 1:10
                x = 1:153;
                y = B(i, :);
                plot(x, y);
                hold on
            end
        \end{lstlisting}    

        \subsection{问题三 神经网络算法 python}
        \begin{lstlisting}[language=python]
            213
        \end{lstlisting}

        \subsection{问题三 神经网络算法 python}
        \begin{lstlisting}[language=python]
            213
        \end{lstlisting}
        
    \end{appendices}
    