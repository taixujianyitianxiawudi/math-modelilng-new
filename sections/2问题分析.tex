\section{问题分析}
    \subsection{问题一}
    问题一为了刻画这些温度时间序列数据的变化情况,我们通过绘图,并且
    观察一些关键变量,如波动趋势,方差,平均值和幅度等来综合判断变化特征
    \subsection{问题二}
    问题二需要对温度数据曲线进行评价。
    在第一题的前提下,从表面的描述变成一个有确定标准的评价:稳定性。
    因此我们需要确定一个评价指标,综合题目的温度与稳定性要求
    我们可以建立一个关于各个参数相关的且带权重函数,将各个温度数据代入到这个
    函数之中。综合各个参数的权重
    比较函数的最终值,从而得出一个最稳定的温度数据。
    \subsection{问题三}
    问题三加入了附件2的数据,使得温度变化规律的数学模型变得更加复杂。
    为了得到一个有价值的数学模型,需要对一些无关或者影响较小的影响因子
    进行舍弃。因此我们通过主成分分析法选取其中权重较大的影响因子
    通过逐步回归模型,确定重要影响因子与系数。比较预测值和实际值的大小,
    确定最终的温度变化规律数学模型
    \subsection{问题四}
    问题四需要我们分析并定位引发超温现象的主要操作变量。这需要我们去
    调整问题三所得到的温度变化模型,调试其中的影响因子,发现其最能影响
    第3147个样本后的走势
    \subsection{问题五}
    问题五需要我们对问题四中发掘出来的重要影响因子进行调优,从而优化
    第十个水冷壁管道的温度曲线。对于目标温度曲线的优化方向,参考问题二
    我们建立的最优目标函数。并且在形态上要与问题二所选定的温度曲线
    趋势大致相同


