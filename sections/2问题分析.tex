\section{问题重述}
    \subsection{问题背景}
    燃煤发电过程中,锅炉起着十分重要的作用。它通过燃煤提高温度并将液态
    水汽化成较高温度的水蒸汽,从而带动汽轮机旋转以达到发电的目的。在家
    用供暖方面,锅炉主要能提供热水和蒸汽。而在工业方面,其主要提供蒸汽
    为其他设备提供制冷、动力等服务。锅炉的主要受热部分是由多排钢管组成
    的水冷壁,内部为用于吸收热量的动态水。
    \subsection{问题提出}
    现根据附件所提供的有关不同时刻不同管道的温度和对应时刻影响水冷壁温度的153个输入变量的数据,建立相
    应的数学模型,并解决以下问题。
    \begin{itemize}
        \item 问题一:分析不同水冷壁管道的温度数据,并对这些温度时间序列
            数据改变情况的特征进行描述。
        \item 问题二:对附件1中10个水冷壁管道的温度数据曲线进行评价,并
            确定其中的最优工作曲线和最差工作曲线。
        \item 问题三: 运用附件1和附件2中的数据,分别建立10个水冷壁管道
            温度变化规律的数学模型,并对其效果进行评价。
        \item 问题四:第10个水冷壁管道温度变化曲线在第3172个样本点后水
            冷壁出现明显的超温现象,根据数据,分析并定位引发超温现象的主要操作变量。
        \item 问题五:针对第10个水冷壁管道温度曲线超温段建立优化模型,
            给出该超温段的最优调节策略,要求操控的变量数尽量少、操作变量的调控量尽量小、优化调节后的工作曲线与问题2中的最优工作曲线的特征尽量吻合。
        
    \end{itemize}


\section{问题分析}
    \subsection{问题一}
    统计分析并刻画不同管道温度时间序列数据的变化情况的特
    征,首先,需要绘制出管道一到管道十的温度时间序列曲线图
    ,并对各条曲线进行初步的观察分析,得到大致的趋势和走向
    。接着,可以计算各管道温度曲线的特征数如,平均值,标准
    差,方差, 最大值和最小值来进一步得到较为准确的温度变化
    特征,并对其进行综合分析。 
    \subsection{问题二}
    首先,该问可以用评价模型去解决,考虑到模型的准确性,摒除
    主观因素的影响,可以选择使用TOPSIS模型处理。对题干进行挖
    掘,可以得到优劣曲线判别的两大指标,从题一中的方差和最大
    温度来加以分析。综合题目的最大温度与稳定性要求我们可以建
    立一个关于各个参数相关的评价模型, 最终可以获得各个管道曲
    线的综合得分,并对得分进行排序,从而判别优劣曲线。
    \subsection{问题三}
    首先考虑到附件2中的输入变量过多,先使用逐步回归法除去与温
    度变化无关的输入变量。为了得到水冷壁管道的变化规律,将剔除
    后的变量与同时刻温度进行多元线性回归,得到各管道温度与剩余
    变量的系数,并通过拟合优度对回归模型效果进行评价。
    \subsection{问题四}
    问题四需要我们分析并定位引发超温现象的主要操作变量。这需要我们去
    调整问题三所得到的温度变化模型,调试其中的影响因子,发现其最能影响
    第3147个样本后的走势
    \subsection{问题五}
    本题需要对管道十温度曲线建立优化模型,可以建立考虑多目标规划
    模型,目标函数要求调整的变量数尽量少、操作变量的改变量尽量小
    且优化调节后的工作曲线与最优工作曲线的特征尽量吻合。通过分
    层序列法,将曲线吻合度放在首位,同时对调整的变量数及变量值
    的改变量赋予相应的权值,最后可以获得管道十优化后的数据,并
    做出相应优化曲线。


